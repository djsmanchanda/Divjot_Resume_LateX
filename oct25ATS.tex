% Document class and font size
\documentclass[a4paper,9pt]{extarticle}

% Packages
\usepackage[utf8]{inputenc} % For input encoding
\usepackage{geometry} % For page margins
\geometry{a4paper, top=0.6in, left=0.4in, right=0.4in, bottom=0.4in}
\usepackage{titlesec} % For section title formatting
\usepackage{enumitem} % For itemized list formatting
\usepackage[hidelinks]{hyperref} % For hyperlinks
\usepackage{textcomp}
\hyphenpenalty=10000
\exhyphenpenalty=10000

% Formatting
\setlist{noitemsep, topsep=4pt, partopsep=4pt} % Removes item separation
\titleformat{\section}{\large\bfseries}{\thesection}{1em}{}[\titlerule] % Section title format
\titlespacing*{\section}{0pt}{\baselineskip}{\baselineskip} % Section title spacing

% Begin document
\begin{document}

% Disable page numbers
\pagestyle{empty}

% Header
\begin{center}
\textbf{\Large DIVJOT SINGH MANCHANDA}\\[6pt]
\href{https://www.djsmanchanda.com}{djsmanchanda.com} | \href{mailto:djsmanchanda@gmail.com}{djsmanchanda@gmail.com} | +91 9137488435 | \href{https://www.linkedin.com/in/djsmanchanda}{linkedin.com/in/djsmanchanda} |  \href{https://github.com/djsmanchanda}{github.com/djsmanchanda} % Contact info
\end{center}

% Experience Section
\section*{EXPERIENCE}

\noindent
\textbf{IIT Madras, CAMS IIT-M Fintech Innovation Lab} \hfill Chennai, Tamil Nadu\\ % Company name and location
\textit{Research Intern} \hfill June 2024 – August 2024 % Position and duration
\begin{itemize}
    \item Fine-tuned a RAG-based LLM trained on financial datasets with web-search capabilities, improving retrieval precision by 30\%.
    \item Conducted feasibility analysis across multiple AI-driven financial models to assess LLM integration in fintech workflows.
\end{itemize}

\noindent
\textbf{Renault Nissan} \hfill Chennai, Tamil Nadu\\ % Company name and location
\textit{Technical Intern} \hfill September 2024 – November 2024 % Position and duration
\begin{itemize}
    \item Engineered an algorithm evaluating 200+ vendors on pricing, lead time, and quality, reducing decision-making time by \textbf{20\%}.
    \item Developed a chatbot for structured access to vendor analytics, cutting manual lookup time by \textbf{40}\%.
\end{itemize}

\noindent
\textbf{FlowNow} \hfill Berkeley, California\\ % Company name and location
\textit{Technical Intern} \hfill March 2025 – June 2025 % Position and duration
\begin{itemize}
    \item Designed an \textbf{agentic AI system} generating context-aware questions from user-submitted books to assess comprehension.
    \item Modeled user attention spans via behavioral metrics, providing real-time feedback and engagement analytics.
    \item Integrated a Tawk.to-powered support agent, reducing onboarding response times by 35\%.
\end{itemize}

% Education Section
\section*{EDUCATION}

\noindent
\textbf{SRM Institute of Science and Technology}, Kattankulathur, Tamil Nadu \hfill September 2022 – June 2026\\ % University name and location
Bachelor of Technology in \textbf{Artificial Intelligence} \hfill \textbf{CGPA: 8.54/10}\\[0.15em]
Relevant Coursework: Neural Networks and Machine Learning, Deep Learning Techniques, Design in Artificial Intelligence Products, Inferential Statistics and Predictive Analytics, Advanced Calculus, Image and Video Processing, Reinforcement Learning.

\vspace{0.5em}\noindent
\textbf{University of California, Berkeley}, Berkeley, California \hfill January 2025 – May 2025\\ % University name and location
\textit{Semester Abroad Program – SCET Startup Semester} \hfill \textbf{GPA: 4.0/4.0}\\[0.15em]
Courses: Deep Learning for Visual Data, Future of Technology \textit{(Designing Next-Generation Technologies)}, Amazoogle \textit{(AI and the Modern Data-Driven Business Model)}, Challenge Lab \textit{(Designing Startups to Transform Society)}, A. Richard Newton Lecture Series.

% Projects Section
\section*{PROJECTS}

\noindent
\textbf{Youtrition} \hfill UC Berkeley, California\\ % Project or organization name and location
\textit{Project Link:} \url{https://github.com/djsmanchanda/Youtrition} \hfill March 2025 – May 2025 % Position and duration
\begin{itemize}
    \item Developed an AI-powered meal recommendation system using the Gemini API to suggest recipes based on users’ dietary restrictions, cuisine preferences, religious beliefs, and daily caloric and protein needs.
    \item Integrated a fridge-vision module using YOLO, SAM, and Gemini to identify ingredients from uploaded photos and recommend suitable recipes.
    \item Provided an interactive interface offering step-by-step recipe guidance and smart ingredient substitutions based on fridge inventory; \textbf{awarded the Tech Maven Award at UC Berkeley}.
\end{itemize}

\vspace{0.2em}
\noindent
\textbf{Reflection (Llama Stack Hackathon)} \hfill Berkeley, California\\ % Project or organization name and location
\textit{Project Link:} \url{https://github.com/djsmanchanda/Reflection} \hfill May 2025 – June 2025 % Position and duration
\begin{itemize}
    \item Built an \textbf{agentic personal assistant} parsing emails, calendars, and chats to autonomously manage scheduling and correspondence.
    \item Integrated \textbf{Qdrant} with Gemini embeddings to store contextual memory, enabling personalized email drafts by tone and style.
    \item Allowed users to review, reprompt, or directly send generated drafts, ensuring transparency and control in communication.
\end{itemize}

\vspace{0.2em}
\noindent
\textbf{Vogue Fusion (PULSE Hackathon)} \hfill IIT Bombay\\ % Project or organization name and location
\textit{Project Link:} \url{https://github.com/djsmanchanda/VogueFusion_Pulse} \hfill November 2024 – December 2024 % Position and duration
\begin{itemize}
    \item Secured \textbf{2nd place} among 500 teams at IIT Bombay’s PULSE Hackathon, earning INR 1L in prize money.
    \item Created ComfyUI pipelines leveraging SDXL and LoRAs to automate design workflows, cutting iteration time by \textbf{50\%}.
    \item Designed a genetic algorithm predicting dye costs within a \textbf{7\%} margin of error for Aditya Birla Group’s fashion lines.
\end{itemize}

\vspace{0.2em}
\noindent
\textbf{MirrorBot} \hfill Intel Unnati AI Labs\\ % Project or organization name and location
\textit{Project Link:} \url{https://github.com/djsmanchanda/MirrorBot} \hfill August 2024 – December 2024 % Position and duration
\begin{itemize}
    \item Processed thousands of Instagram and WhatsApp messages to train a model replicating user conversational tone.
    \item Fine-tuned LLaMA 3.2 (7B) on curated datasets to achieve \textbf{92\%} stylistic similarity in conversational responses.
    \item Conducted as part of the \href{https://drive.google.com/file/d/19qXnH3R2kTSqghv5wdkE-2p2tb6FoV1I}{Intel Unnati Generative AI Industrial Training – 2024}.
\end{itemize}

% Skills Section
\section*{SKILLS}
\begin{itemize}
    \item \textbf{Programming Languages:} Python, C, C++, CUDA, PowerShell, Bash
    \item \textbf{Frameworks \& Libraries:} PyTorch, TensorFlow, Keras, Hugging Face, LangChain, Flask, Streamlit, React, Next.js, ComfyUI
    \item \textbf{Tools and Platforms:} Git, Docker, AWS
\end{itemize}

% Achievements
\section*{ACHIEVEMENTS}
\noindent
\textbf{Hackathons:}
Won 7+ hackathons and competitions with cumulative cash prizes exceeding \textbf{INR 2,00,000 and USD 1,000}.
\begin{itemize}
    \item \textbf{1st place} – HackStreet 2.0, SRM, Chennai \hfill March 2024
    \item \textbf{2nd place} – Llama Stack Challenge, UC Berkeley \hfill April 2025
    \item \textbf{2nd place} – PULSE Hackathon, Techfest IIT Bombay \hfill December 2024
    \item \textbf{3rd place} – Google Cloud Hacknight, San Francisco \hfill May 2025
\end{itemize}

\noindent
\textbf{Line Follower Competitions:}
Built a \textbf{high-speed, high-precision LFR} using a custom PCB and PID control algorithm.
\begin{itemize}
    \item \textbf{1st place} – BITS Pilani \hfill March 2023
    \item \textbf{2nd place} – Kongu Engineering College \hfill February 2023
    \item \textbf{3rd place} – Anna University, Chennai \hfill April 2023
\end{itemize}

\noindent
\textbf{Research Recognition:}
\begin{itemize}
\item Awarded a \textbf{Dean-Funded Research Grant} for developing an in-house lift optimization device that reduced queue times and increased hourly capacity by \textbf{7\%} through AI-driven control.

\end{itemize}



% Volunteering Section
\section*{VOLUNTEERING – Student Clubs}

\noindent
\textbf{Data Science Community} \hfill \textit{Machine Learning / Research Supervisor}\\
\textit{Developed a transformer-based music generator and mentored peers in applied ML research.}\\[2pt]

\noindent
\textbf{SRM Team Robocon} \hfill \textit{System Integration and Electronics Engineer}\\
\textit{Gained 1,000+ hours of hands-on experience in control systems, embedded coding, and robot integration.}\\[2pt]

\noindent
\textbf{CINTEL’s Next Gen AI} \hfill \textit{Domain Head – Robotics and Generative AI}\\
\textit{Led robotics and generative AI initiatives, mentoring 25 students and organizing hackathons with 1,000+ participants.}

% End document
\end{document}